%----------------------------------------------------------------------------------------
%	Onderwijs
%----------------------------------------------------------------------------------------
% \hfil{\Large{\bf Appendix - Onderwijs}}\hfil\\
%------------------------------------------------
\begin{rSection}{Didactische profesionalisering}
\begin{rSubsection}{Basiskwalificatie Onderwijs (BKO)}{februari 2024 - heden}{OnderwijsCentrum UMC Utrecht}{Utrecht, NL}
  \vspace{-1.5em}
\end{rSubsection}
\end{rSection}

%------------------------------------------------
\begin{rSection}{Onderwijsrollen}
\begin{rSubsection}{Co\"{o}rdinator "DevOps for Health Data Science (3 EC)"}{2024 - heden}{Graduate School of Life Science (GSLS)}{Utrecht, NL}
  \vspace{-1.5em}
\end{rSubsection}

\begin{rSubsection}{Scriptiebegeleider}{2017 - heden}{}{}
    \vspace{-1.5em}
    \item Niels Helmantel, 2024, UU, Utrecht - M.Sc. scriptie HCI "Design of an Explainable Interface for a Sepsis Prediction Algorithm" (met H. Hauptmann \& D. Vijlbrief)
    \item Selin Acan, 2023, UvA, Amsterdam - M.Sc. scriptie software engineering "Prediction of the Outcome of Asphyxia Patients After Hypothermic Treatment with Continuous Variables" (met T. Alderliesten)
    \item Annie Clarnette, 2022, UU, Utrecht - M.Sc. scriptie ADS "Explainable machine learning algorithm for early prediction of late-onset sepsis in preterm infants using real-time catheter data" (met D. Vijlbrief)
    \item Felix Bindt, 2022, UU, Utrecht - M.Sc. scriptie ADS "Predicting Late-Onset Sepsis using Machine Learning with a Minimum Feature, Blood Pressure Centered, Clinic Oriented Approach" (met D. Vijlbrief)
    \item Joshua Oosterlaken, 2022, UU, Utrecht - M.Sc. scriptie ADS "The prediction of late-onset sepsis using generalized additive models" (met D. Vijlbrief)
    \item Kathleen Short, 2017, UvA, Amsterdam - M.Sc. scriptie physics "Millisecond Pulsars and the Galactic Centre GeV Excess: Prospects for e-ASTROGAM" (met C. Weniger) 
  \end{rSubsection}

\end{rSection}

%------------------------------------------------
\begin{rSection}{Overige Onderwijservaring}

\begin{rSubsection}{Onderwijsassistent}{2015 - 2017}{}{}
  \vspace{-1.5em}
	\item Cosmology (M.Sc. cursus), najaar 2017, Universiteit van Amsterdam
	\item General relativity (M.Sc. cursus), voorjaar 2017, Universiteit van Amsterdam
	\item Particles and fields (M.Sc. cursus, voorjaar 2016, Universiteit van Amsterdam
	\item Astroparticle physics (mM.Sc. cursus), voorjaar 2015, Universiteit van Amsterdam
	\item Advanced statistics (M.Sc. cursus), winter 2015, Universiteit van Amsterdam
  \item Natuur- en wiksunce, september 2011 - juni 2012, University College Utrecht 
\end{rSubsection}

\begin{rSubsection}{Invaldocent natuur- en wiskunde (HAVO 4/5 en VWO 5/6)}{}{KlasseStudent}{Utrecht, NL}
  \vspace{-1.5em}
\end{rSubsection}

\begin{rSubsection}{Outreach}{}{}{}
  \vspace{-1.5em}
      \item Lezing - "Wat is donkere materie? De grote onopgeloste puzzel in de moderne (astro)fysica" - Volkssterrenwacht Orion,  Bovenkarspel (9 januari 2019)
      \item Assistent bij Altair (kennismaking astronomie en natuurkunde voor basisschoolleerlingen) - groep 7 As-Siddieq Oost, Amsterdam (maart - april 2018)
\end{rSubsection}
\end{rSection}

%------------------------------------------------

%----------------------------------------------------------------------------------------
