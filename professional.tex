%----------------------------------------------------------------------------------------
%	Academic Appendix
%----------------------------------------------------------------------------------------
\hfil{\Large{\bf Appendix - Professional Experience}}\hfil\\
\nameskip\break

%------------------------------------------------
\begin{rSection}{Online certificates}
%------------------------------------------------
  \begin{itemize}
    \item{\it AI for Medical Diagnosis} -- deeplearning.ai on Coursera.
    \hef{https://coursera.org/share/aea6e47d93de58241709766351f714aa}{Certificate} earned on October 30, 2020.
    %
    \item{\it Building a Data Science Team} -- Johns Hopkins University on Coursera.
    \hef{https://coursera.org/share/9d296dc31002fa58759569a1e2ebcddc}{Certificate} earned on July 14, 2020.
    %
    \item{\it Taming Big Data with Apache Spark and Python - Hands On!} -- by Sundog Education on Udemy. \href{https://www.udemy.com/certificate/UC-G53DWS5V/}{Certificate} earned on April 15, 2019.
    %
    \item {\it Practical Reinforcement Learning} -- National Research University Higher School of Economics on Coursera. \href{https://coursera.org/share/7dbaef1de4f13d7016316a4dd7eece8a}{Certificate} earned on February 14, 2019.
    %
    \item{\it Deep Learning, a 5-course specialization} -- deeplearning.ai on Coursera. \href{https://coursera.org/share/47b1d9ff10e34c9d26c75f5f0914379b}{Certificate} earned on January 7, 2019.
    %
    \subitem {\it Neural Networks and Deep Learning} -- deeplearning.ai on Coursera. \href{https://coursera.org/share/b99cfdbf684af876690916f62a41f74d}{Certificate} earned on December 11, 2018.
    %
    \subitem {\it Improving Deep Neural Networks: Hyperparameter tuning, Regularization and Optimization} -- deeplearning.ai on Coursera. \href{https://coursera.org/share/70436cea6ad8695c56da96f285739488}{Certificate} earned on December 21, 2018.
    %
    \subitem {\it Structuring Machine Learning Projects} -- deeplearning.ai on Coursera. \href{https://coursera.org/share/b4fec0b5c8ada28277e0369aced10c34}{Certificate} earned on December 24, 2018.
    %
    \subitem {\it Convolutional Neural Networks} -- deeplearning.ai on Coursera. \href{https://coursera.org/share/efaf45dddda63778f97da9aafaaa71b0}{Certificate} earned on December 30, 2018.
    %
    \subitem {\it Sequence Models} -- deeplearning.ai on Coursera. \href{https://coursera.org/share/aeb503aafcd63fd9b1e098c232c6cfa4}{Certificate} earned on January 7, 2019.
    %
    \item {\it Google Cloud Platform Big Data and Machine Learning Fundamentals} -- by Google Cloud on Coursera. \href{https://coursera.org/share/469ce1832e0392458bc6930beb44d8c6}{Certificate} earned on October 1, 2018.
  \end{itemize}
\end{rSection}

%%------------------------------------------------
%\begin{rSection}{Further information: work experience}
%%------------------------------------------------
%\begin{rSubsection}{Sensing Clues}{December 2019 - Present}{Data Scientist (through Vantage AI)}{Nieuwegein, The Netherlands}
%Sensing Clues is a not-for-profit organisation which offers an analysis platform for wildlife parks throughout the world.
%Information is collected by rangers through an (offline) app, which connects to the platform.
%I did the acquisition and as part of the Vantage for Social Good initiative I was given limited time to work pro-bono for Sensing Clues.
%
%My role is to orchestrate projects related to computer vision and help identify use cases.
%A first use case we started is the automatic detection of animals in cameratraps, such that this information can be efficiently included in the platform.
%Together with a colleague I created a single algorithm which can be used for parks throughout the world.
%The algorithm was developed using data from the \href{https://www.kaggle.com/c/iwildcam-2020-fgvc7/leaderboard}{iWildCam 2020} competition on Kaggle,
%where we ended in the top 5.
%Currently we are in the process of taking the first version to production.
%
%{\bf Skills}: object detection, use-case discovery\\
%{\bf Tools}: TensorFlow, TensorFlow Extended, Flask, Kubernetes, Docker
%\end{rSubsection}
%
%%------------------------------------------------
%\begin{rSubsection}{ProRail}{May 2019 - Present}{Data Scientist (through Vantage AI)}{Utrecht, The Netherlands}
%ProRail is the owner and maintainer of the railway infrastructure in the Netherlands.
%Twice a year, a videotrain captures images of the entire railway network of the Netherlands, from 10 different
%camera perspectives.
%Just the two cameras focused on the top of the railway track already take 50 million images during each campaign.
%These images are inspected manually to find defects. However, it is impossible to manually assess all images.
%As such, ProRail has the ambition to automate certain tasks by means of computer vision.
%At first, they want to locate all insulation joints in the Netherlands and monitor and correct the location information in
%their SAP database. Next, it is their goal to also look at the condition and degeneration of various assets.
%
%The project entailed creating a convolutional neural network to detect insulation-joints in images and putting it in production
%This project had high business impact, since malfunctioning insulation joints cause a large fraction of signalling failures,
%so accurate knowledge of their location is essential. Next to developing the algorithm, I was involved in setting up a
%production pipeline for which a package was created to read-in images, images are analysed, and finally results
%are included in the asset database of ProRail.
%
%In addition, I created an object detection algorithm to supplement the aforementioned model,
%which adds additional information on each individual joint. Moreover, this algorithm also forms the basis
%for a proof of concept (PoC) to detect damaged railway fasteners.
%
%{\bf Skills}: computer vision, object detection, data science in production, stakeholder management, use-case discovery\\
%{\bf Tools}: TensorFlow, Azure Databricks, Azure Storage (Datalake, Blob), OpenCV
%\end{rSubsection}
%
%%------------------------------------------------
%\begin{rSubsection}{Intergamma}{February 2019 - March 2019}{Data Scientist (through Vantage AI)}{Leusden, The Netherlands}
%Intergamma is a large franchise-organisation for hardware stores.
%Intergamma has an online shop, but it operates mostly through physical stores.
%At the time of the assignment Intergamma was applying data science within a limited scope, namely e-commerce.
%However, they had the ambition to become more data-driven.
%
%To explore the potential of data science for the organisation,
%a colleague and I were assigned the task to perform a use-case discovery, in particular outside of e-commerce.
%For this purpose I conducted interviews with various stakeholders and domain experts, and looked at data availability.
%
%In addition, two use cases were used to realise a proof-of-concept.
%First, I have done a market-basket analysis to optimise the in-store shelf layout.
%For this purpose I analysed online and offline transactions by means of similarity matrices of products and product categories.
%By comparing online to offline cross-sell the shelf layout can potentially be further optimised.
%The second PoC was about price elasticity and promotions and focused on the feasibility of optimising the optimal price
%with machine learning models.
%
%{\bf Skills}: use-case discovery, proof-of-concept, strategy\\
%{\bf Tools}: Python, Pandas, Google Cloud Platform, SQL (Netezza)
%%
%\end{rSubsection}
%%------------------------------------------------
%\begin{rSubsection}{Stockon}{November 2018 - December 2018}{Data Scientist (through Vantage AI)}{Utrecht, The Netherlands}
%%
%Stockon (discontinued) was an online supermarket operating through an app and delivering groceries through Post NL.
%One of their key selling points was to ease the process of doing groceries, by filling the shopping basket
%for individual customers automatically through an algorithm that would predict individual consumption.
%However, initially this algorithm could only be used for the 15\% of customers who had a subscription for recurring orders.
%Moreover, it was implemented in TypeScript and not easy to update or maintain.
%I was asked to make the algorithm applicable to all customers and create a microservice in the Google Cloud within a limited timespan.
%
%For this project I was the only data scientist, as such I sought active collaboration with the software developers responsible for ensuring functionality within the app.
%Together we devised a strategy to migrate the prediction algorithm to the cloud.
%I extended the existing algorithm, rewrote it in Python and created a microservice through a Flask API running
%in a Docker container in the Google App Engine. The result was a minimal-viable product serving
%100\% of customers by updating their shopping baskets on a daily basis.
%
%{\bf Skills}: data science in production, data engineering, agile and scrum methodology\\
%{\bf Tools}: Python, Pandas, Flask, Docker, Google App Engine.
%%
%\end{rSubsection}
%%------------------------------------------------
%\begin{rSubsection}{Winkelstraat.nl}{September 2018 - October 2018}{Data Scientist (through Vantage AI)}{Vianen, The Netherlands}
%%
%Winkelstraat.nl is a platform which connects customers to brick-and-mortar retailers of high-end fashion items by
%offering the latter an opportunity to sell their products online. Consequently, Winkelstraat.nl gathers large amounts of
%data which could be of potential use to the affiliated retailers. It was my task to identify use-cases based on existing
%and potential data sources.
%
%For this project I connected multiple existing data sources and established key performance indicators (KPIs) for individual retailers.
%In addition, I suggested data sources which could help to further refine these KPIs.
%
%The results where presented in an interactive dashboard (Google Data Studio) tailored to individual users.
%
%{\bf Skills}: strategy, business intelligence, data visualisation and dashboarding, data science\\
%{\bf Tools}: Python, Pandas, Jupyter, Docker Google Cloud Platform, Google BigQuery, Google Data Studio
%%
%\end{rSubsection}
%\end{rSection}
%----------------------------------------------------------------------------------------

% Talks 
\begin{rSection}{Talks}
  \begin{itemize}
  	\begin{rSubsection}{Public talks}{}{}{}
       	\item{\it Next Gen Railway Asset Management} (with Jasper Derikx) - PyData Eindhoven - Eindhoven, the Netherlands (29-30 November 2019)
	\item{\it Next Gen Railway Asset Management} (with Oscar van Hees) - Big Data Expo - Jaarbeurs, Utrecht, the Netherlands (18-19 September 2019) - 
	\href{https://www.bigdata-expo.nl/nl/infographic-programma-2019}{Top 10 best talks}
	\end{rSubsection}
	 %
  	\begin{rSubsection}{Podcasts}{}{}{}
       	\item De Dataloog - 
   	\href{https://www.dedataloog.nl/uitzending/dtl-big-data-expo-special-8-het-belang-van-onzekerheid-bij-ml-modellen/}{Het belang van onzekerheid bij ML modellen} (with Lieke Kools).
	\end{rSubsection}
	%
  \end{itemize}
\end{rSection}

%% Talks
%\begin{rSection}{Workshops}
%  \begin{itemize}
%  	\item Computer vision with Tensorflow (with Joris Bukala) - Vantage AI (May - June 2020)
%       	\item Bayesian methods in machine learning  (with Gerben Oostra) - Vantage AI (November - December 2019)
%	\item Computer vision with Tensorflow 2.0 - Vantage AI (July - August 2019)
%  \end{itemize}
%\end{rSection}
